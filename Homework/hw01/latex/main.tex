\documentclass[a4paper,12pt,twoside]{article}

\usepackage{preamble}

\title{Homework 01}
\author{Brandon Henke\\PHY831\\Scott Bogner}
\date{September 17, 2021}

\setcounter{section}{1}

\fancyhead[LE,RO]{B. Henke}
\fancyhead[RE,LO]{\thepage}

\begin{document}
%\tableofcontents

\maketitle
\begin{multicols*}{2}


\subsection{}%1
\subsubsection*{a}
From the first law,
\begin{equation}
	dQ = \dd{E} - dW.
\end{equation}
Hence,
\begin{align}
	Q_h &= \int \left( \pdv{E}{\Theta} \dd{\Theta} +  P \dd{V}\right),\\
	&= \int \frac{Nk_B\Theta_h}{V}\dd{V},\\
	&= Nk_B\Theta_h \ln(\frac{V_2}{V_1}).
	\label{eq: 1a Q_h}\\
	Q_c &= Nk_B\Theta_c \ln(\frac{V_3}{V_4}).
	\label{eq: 1a Q_c}
\end{align}
For \ref{eq: 1a Q_c}, heat is being expelled, hence the minus sign (evident in swapping the bounds of integration).
\subsubsection*{b}
From the first law,
\begin{align}
	\dd{E} &= -P\dd{V},\\
	\pdv{E}{\Theta}\dd{\Theta} &= -P\dd{V}.
	\label{eq: 1b_first_law}
\end{align}
Assuming an ideal gas,
\begin{equation}
	P = \frac{Nk_B \Theta}{V}.
\end{equation}
Plugging this into \ref{eq: 1b_first_law} gives
\begin{equation}
	\frac{1}{\Theta} \pdv{E}{\Theta} \dd{\Theta} = -Nk_B \frac{\dd{V}}{V}.
\end{equation}
Integrating both sides from $(V_0,\Theta_0)$ to $(V,\Theta)$ gives
\begin{equation}
	\int_{\Theta_0}^\Theta \frac{1}{\tau} \left.\pdv{E}{\Theta}\right|_\tau \dd{\tau} = -Nk_B \ln(\frac{V}{V_0}),
\end{equation}
or
\begin{equation}
	\frac{V}{V_0} = e^{-\frac{1}{Nk_B}\int_{\Theta_0}^\Theta \frac{1}{\tau} \left.\pdv{E}{\Theta}\right|_\tau \dd{\tau}}.
	\label{eq: 1b_answer}
\end{equation}

\subsubsection*{c}
By definition,
\begin{equation}
	\eta \equiv \frac{W}{Q_h} = \frac{Q_h-Q_c}{Q_h} = 1-\frac{Q_c}{Q_h}.
\end{equation}
From equations \ref{eq: 1a Q_h} and \ref{eq: 1a Q_c},
\begin{equation}
	1-\frac{Q_c}{Q_h} = 1-\frac{\Theta_c \ln(V_3/V_4)}{\Theta_h \ln(V_2/V_1)}
\end{equation}
Additionally, from \ref{eq: 1b_answer}, $\ln(V_2/V_1) = \ln(V_3/V_4)$.
Thus
\begin{equation}
	\eta = 1- \frac{\Theta_c}{\Theta_h}.
\end{equation}

\subsection{}%2
\subsubsection*{a}
Take the first law:
\begin{equation}
	dQ = \dd{E} - dW.
\end{equation}
If one uses $(T,V)$ as the independent variables, then
\begin{align}
	dQ &= \pdv{E}{T} \dd{T} + \left( \pdv{E}{V} + P \right) \dd{V},\\
	&= C_V\dd{T} + \left( \pdv{E}{V} + P \right) \dd{V}.
\end{align}
\subsubsection*{b}
Finding an expression for $\dd{S}$, if follows from the second law that
\begin{align}
	\dd{S} &= \frac{dQ}{T},\\
	&= \frac{C_V}{T}\dd{T} + \left( \frac{1}{T}\pdv{E}{V} + \frac{P}{T} \right) \dd{V}.
\end{align}
By definition of an exact differential,
\begin{align}
	\pdv{S}{V} &= \frac{1}{T}\pdv{E}{V} + \frac{P}{T},\\
	\pdv{S}{T} &= \frac{C_V}{T}.
\end{align}
Additionally
\begin{equation}
	\pdv{S}{V}{T} = \pdv{S}{T}{V},
\end{equation}
so
\begin{align}
	\pdv{V} \frac{C_V}{T} &= \pdv{T}(\frac{1}{T}\pdv{E}{V} + \frac{P}{T}),\\
	0 &= -\frac{1}{T^2}\pdv{E}{V} - \frac{P}{T^2} + \frac{1}{T}\pdv{P}{T},\\
	&= \pdv{E}{V} + P - T\pdv{P}{T},\\
	\pdv{E}{V} &= T\pdv{P}{T} - P.
\end{align}

\subsubsection*{c}
For an ideal gas,
\begin{equation}
	\pdv{E}{V} = \frac{Nk_BT}{V} - P = 0.
\end{equation}
Thus the internal energy cannot depend on $V$, and is only a function of $T$.
However, since no consideration was made pertaining to the particle number, $N$, the internal energy could still depend on $N$.
Therefore $E = E(T,N)$.
\subsection{}%3
From the first law,
\begin{equation}
	\dd{E} = -P\dd{V} - S\dd{T} + \mu \dd{N},
\end{equation}
where
\begin{align}
	P   &\equiv -\pdv{E}{V},\\
	S   &\equiv -\pdv{E}{T},\\
	\mu &\equiv  \pdv{E}{N}.
\end{align}
The Maxwell relations mentioned are derived as follows:
\begin{align}
	\pdv{\mu}{T} &= \pdv{E}{T}{N} = -\pdv{S}{N}.\\
	\pdv{S}{V} &= -\pdv{E}{T}{V} = \pdv{P}{T}.
\end{align}
Inverting these gives the requested relations:
\begin{align}
	\pdv{T}{\mu} &= -\pdv{N}{S},\\
	\pdv{P}{T} &=\pdv{S}{V}.
\end{align}
\subsection{}%4
\subsubsection*{a}
The Helmholtz Free Energy is given by
\begin{align}
	A-A_0 &= T_0 \ln(\frac{V}{V_0}) \nonumber \\
	&-\frac{1}{a+1} \frac{V}{V_0}\left( \frac{T^{a+1}}{T_0^{a}} - T_0 \right).
\end{align}
\subsubsection*{b}
The equation of state is found by using the Maxwell relation
\begin{align}
	-P &= \pdv{A}{V},\\
	&= \frac{T_0}{V} - \frac{1}{a+1} \frac{1}{V_0}\left( \frac{T^{a+1}}{T_0^{a}} - T_0 \right).
	\label{eq: 4b_answer}
\end{align}

\subsubsection*{c}
From \ref{eq: 4b_answer},
\begin{align}
	W &= -\int_{V_a}^{V_b} P \dd{V},\\
	&= \int_{V_a}^{V_b} \frac{T_0}{V}\dd{V} \nonumber\\
	&- \int_{V_a}^{V_b}\frac{1}{a+1} \frac{1}{V_0}\left( \frac{T_a^{a+1}}{T_0^{a}} - T_0 \right) \dd{V},\\
	&= T_0\ln(\frac{V_b}{V_a}) \nonumber \\
	&- \frac{1}{a+1}\frac{(V_b-V_a)T_a}{V_0}\left( \frac{T_a^{a+1}}{T_0^{a}} - T_0 \right).
\end{align}

\nocite{*}
\printbibliography[title={References},heading=bibnumbered]


\end{multicols*}
\end{document}
