\documentclass[a4paper,12pt,twoside]{article}

\immediate\write18{echo -n "\\newcommand{\\gitroot}{" > gitroot.txt && git rev-parse --show-toplevel >> gitroot.txt && truncate -s-1 gitroot.txt && echo -n "}" >> gitroot.txt}

\input{gitroot.txt}
\usepackage{\gitroot/latexTemplate/preamble}


\newcommand{\Author}{Brandon Henke}
\newcommand{\course}{\textit{PHY831}}
\newcommand{\professor}{\textit{Scott Bogner}}

\newcommand{\mcols}{0}


\title{Homework 04}
\author{
	Brandon Henke\\
	\course\\
	\professor
}
\date{Ottobre 15, 2021}


\fancyhead[LE,RO]{B. Henke}
\fancyhead[RE,LO]{\thepage}

\bibSetup{refs.bib}

\begin{document}
%\tableofcontents

\maketitle
\if\mcols1
\begin{multicols*}{2}
\fi

\setcounter{section}{4}
\subsection{}%1
\subsubsection{}
\begin{align}
	\mathcal{E}_{max} \leq &\mathcal{S} \leq \mathcal{N} \mathcal{E}_{max},\\
	\frac{1}{N}\log(\mathcal{E}_{max}) \leq \frac{1}{N}&\log(\mathcal{S}) \leq \frac{1}{N}\log(\mathcal{N} \mathcal{E}_{max}),\\
	\varphi_{max} \leq \frac{1}{N}&\log(\mathcal{S}) \leq \frac{1}{N}\log(\mathcal{N}) + \varphi_{max}.
\end{align}
\subsubsection{}
In the limit as $N\rightarrow\infty$, $\mathcal{N} \rightarrow N^p$.
Hence
\begin{equation}
	\lim_{N\rightarrow \infty} \frac{1}{N} \log(N^p) = \lim_{N\rightarrow \infty} \frac{p}{N} \log(N) = 0.
\end{equation}
Therefore,
\begin{align}
	\lim_{N\rightarrow\infty} \frac{1}{N}\log(\mathcal{S}) \rightarrow &\varphi_{max}
	\leq \lim_{N\rightarrow\infty} \frac{1}{N}\log(\mathcal{S})
	\leq \varphi_{max}+ \lim_{N\rightarrow\infty} \frac{1}{N}\log(\mathcal{N}),\\
	&\varphi_{max}
	\leq \lim_{N\rightarrow\infty} \frac{1}{N}\log(\mathcal{S})
	\leq \varphi_{max},\\
	&\lim_{N\rightarrow\infty} \frac{1}{N}\log(\mathcal{S}) = \varphi_{max}.
\end{align}
\subsection{}%2
\begin{align}
	\dv{p_i}{t} =& \sum_{j} \omega_{ij}(p_j-p_i).\\
	S =& -\sum_{i} p_i\log(p_i),\\
	\dv{S}{t} =& -\sum_{i}\dv{p_i}{t}\log(p_i) - \dv{t}(\sum_{i}p_i),\\
	=& -\sum_{i}\dv{p_i}{t}\log(p_i),\\
	=& \sum_{i,j} \omega_{ij}(p_i-p_j)\log(p_i).
\end{align}
If $p_i = p_j$, $\dv*{S}{t} = 0$.
Thus,
\begin{align}
	\dv{S}{t} =& \sum_{i,j} \omega_{ij}\frac{1}{2}\left((p_i-p_j)\log(p_i)-(p_i-p_j)\log(p_j)\right),\\
	=& \sum_{i,j} \frac{\omega_{ij}}{2}(p_i-p_j)\log(\frac{p_i}{p_j}).
\end{align}
If $p_i > p_j$, then $(p_i-p_j) > 0$ and $\log(\frac{p_i}{p_j}) > 0$.
Lastly, if $p_j > p_i$, then $(p_i-p_j) < 0$ and $\log(\frac{p_i}{p_j}) < 0$, so their product is positive.
Therefore,
\begin{equation}
	\dv{S}{t} \geq 0.
\end{equation}
\subsection{}%3
\subsubsection{}
The definition of the grand partition function is
\begin{equation}
	\mathcal{Z} = \sum_{N=0}^\infty z^N Z_N,
	\label{eq: gce}
\end{equation}
where $z = e^{\beta\mu}$ is the fugacity, and $Z_N$ is the canonical partition function for the system with $N$ particles.
\begin{align}
\mathcal{Z}(T,\mu,V)
=& \sum_{N=0}^\infty e^{\beta \mu N} \frac{V^N}{N!(2\pi \hbar)^{3N}} \left(\int \dd[3]{\vb{p}} e^{-\beta \frac{p^2}{2m}}\right)^N,\\
=& \sum_{N=0}^\infty e^{\beta \mu N} \frac{V^N}{N!} \left(\frac{m}{2\pi \hbar^2 \beta}\right)^{3N/2}.
\end{align}
Let
\begin{equation}
	l_Q = \sqrt{\frac{2\pi \hbar^2 \beta}{m}}.
\end{equation}
Then
\begin{equation}
	\mathcal{Z}(T,\mu,V) = \sum_{N=0}^\infty  \frac{e^{\beta \mu N}}{N!} \frac{V^N}{l_Q^{3N}} = \sum_{N=0}^\infty  \frac{z^N}{N!} \frac{V^N}{l_Q^{3N}} = e^{z V / l_Q^3}.
	\label{eq: p3_gpf}
\end{equation}
\subsubsection{}
\begin{equation}
	e^{-\beta \mathcal{G}} = \mathcal{Z},
\end{equation}
where $\mathcal{G}$ is the grand potential.
Solving for $\mathcal{G}$ gives
\begin{equation}
	\mathcal{G}(T,\mu,V) = -\frac{1}{\beta} \log(\mathcal{Z}) = -\frac{e^{\beta \mu}V}{\beta l_Q^3}.
\end{equation}
By definition of pressure,
\begin{align}
	P =& -\pdv{\mathcal{G}}{V},\\
	=& \frac{e^{\beta \mu}}{\beta l_Q^3}.
\end{align}
Additionally,
\begin{align}
	N = \ev{N} =& - \pdv{\mathcal{G}}{\mu},\\
	=& \frac{e^{\beta \mu}V}{l_Q^3}.\label{eq: p3_evN}
\end{align}
Hence
\begin{align}
	PV =& \frac{e^{\beta \mu}}{\beta l_Q^3}\frac{l_Q^3 N}{e^{\beta \mu}},\\
	PV =& \frac{N}{\beta}. \qquad\checkmark
\end{align}
\subsubsection{}
The Poisson distribution is defined as
\begin{equation}
	P(k) \equiv \frac{\lambda^k e^{-\lambda}}{k!},
	\label{eq: poisson}
\end{equation}
for some random variable, $X = k$, where $\lambda$ is the expectation value of the random variable.
In the case of the problem, the random variable is $N$, and the expectation value of this variable was found to be $\lambda = \ev{N} = zV/l_Q^3$.

To show that the particle number is described by the Poisson distribution, start from the probability of finding $N$ particles in the system:
\begin{align}
	P(N) = \frac{z^N Z_N}{\mathcal{Z}}
	=& \frac{1}{\mathcal{Z}} \frac{(zZ_1)^N}{N!},\\
	=& \frac{1}{\mathcal{Z}} \frac{e^{\beta\mu}V}{N! l_Q^3}.
\end{align}
Substituting in \ref{eq: p3_gpf} and \ref{eq: p3_evN} gives
\begin{equation}
	P(N) = \frac{\ev{N}}{N!}e^{-z V / l_Q^3} = \frac{\ev{N}e^{-\ev{N}}}{N!}.
\end{equation}
Hence, the distribution of particle numbers is Poissonian.
\subsection{}%4
\subsubsection{}
The Helmholtz free energy, $A$ is given by
\begin{equation}
	A = -\frac{\log Z_N}{\beta}.
\end{equation}
The partition function for the system is given by
\begin{align}
	Z_N =& \frac{1}{N!(2\pi\hbar)^{6N}}\left( \int e^{-\frac{\beta}{2m}(p_1^2+p_2^2)}e^{-\frac{\beta K}{2}\abs{\vb{r}_1-\vb{r}_2}^2}\dd[3]{p_1}\dd[3]{p_2}\dd[3]{r_1}\dd[3]{r_2}\right)^N,\\
	=& \frac{1}{N!(2\pi\hbar)^{3N}} \left(\frac{2\pi m}{\beta}\right)^{3N}\left( \int e^{-\frac{\beta K}{2}\abs{\vb{r}_1-\vb{r}_2}^2}\dd[3]{r_1}\dd[3]{r_2}\right)^N,\\
	=& \frac{1}{N!l_Q^{6N}}\left(\int e^{-\frac{\beta K}{2}\abs{\vb{r}_1-\vb{r}_2}^2}\dd[3]{r_1}\dd[3]{r_2}\right)^N.
\end{align}
Let $\vb{r} = \vb{r}_1 - \vb{r}_2$ and $\vb{R} = \frac{1}{2}(\vb{r}_1 + \vb{r}_2)$.
Then the integral becomes
\begin{align}
	Z_N =& \frac{1}{N!l_Q^{6N}}\left(\int \dd[3]{R}\int e^{-\frac{\beta K}{2}\vb{r}^2}\dd[3]{r}\right)^N,\\
	=& \frac{V^N}{N!l_Q^{6N}} \sqrt{\frac{2\pi}{\beta K}}^{3N}.
\end{align}
The Helmholtz free energy, then, is
\begin{align}
	A = -\frac{1}{\beta}\log{Z_N} =& -\frac{N}{\beta} \log(\frac{V}{N!l_Q^6})-\frac{3N}{2\beta}\log(\frac{2\pi}{\beta K})
\end{align}
\subsubsection{}
By the equipartition theorem,
\begin{equation}
	C_v = \frac{9}{2}N k_B,
\end{equation}
since there are nine different ways to put energy into the molecule:
\begin{enumerate}
	\item three directions of relative translation,
	\item three directions of relative velocities, and
	\item three directions of coupled velocity.
\end{enumerate}
\subsubsection{}
The mean square molecular diameter is given by
\begin{equation}
	\ev{\abs{\vb{r}_1-\vb{r}_2}^2} = \ev{r^2}
	= \frac{\int r^2 e^{-\frac{\beta K}{2}r^2}\dd[3]{r}}{\int e^{-\frac{\beta K}{2}r^2}\dd[3]{r}}.
\end{equation}
One notices that the numerator is given by
\begin{align}
	\int r^2 e^{-\frac{\beta K}{2}r^2}\dd[3]{r} =& -2\pdv{(\beta K)}\left( \int r^2 e^{-\frac{\beta K}{2}r^2}\dd[3]{r} \right),\\
	=& -2\pdv{(\beta K)} \sqrt{\frac{2\pi}{\beta K}}^3,\\
	=& \frac{3}{2}\sqrt{\frac{(2\pi)^3}{(\beta K)^5}}.
\end{align}
Hence,
\begin{equation}
	\ev{r^2} = \frac{3}{2}\sqrt{\frac{(2\pi)^3}{(\beta K)^5}} \sqrt{\frac{\beta K}{2\pi}}^3
	= \frac{3}{\beta K}.
\end{equation}
\subsection{}%5
\subsubsection{}
\begin{align}
	\ev{A}{\Psi} =& \left(\sum_{\beta,j}C_{\beta j}^* \bra{\beta}\otimes\bra{j}\right) \left(\sum_{\alpha,i}C_{\alpha i} (A\ket{\alpha})\otimes\ket{i}\right),\\
	=& \sum_{\alpha,\beta,i,j} C_{\beta j}^*C_{\alpha i} \mel{\beta}{A}{\alpha}\otimes\ip{j}{i},\\
	=& \sum_{\alpha,\beta,i} C_{\beta i}^*C_{\alpha i} \mel{\beta}{A}{\alpha},\\
	=& \sum_{\alpha,\beta} \rho_{\alpha\beta} \mel{\beta}{A}{\alpha},\\
	=& \Tr_s\left(A\rho\right).\\
\end{align}
\subsubsection{}
The density operator is Hermitian:
\begin{align}
	\rho^\dagger= \rho_{\beta\alpha}^* =& \sum_{i} C_{\beta i}^*C_{\alpha i},\\
	=& \sum_{i} C_{\alpha i}C_{\beta i}^*,\\
	=& \rho_{\alpha\beta}.
\end{align}
Since $\rho_{\alpha\beta}$ is hermitian, it can be written as a linear combination of eigenvectors and their respective eigenvalues:
\begin{align}
	\rho_{\alpha\beta} =& \mel{\alpha}{\left(\sum_k \lambda_k \op{\rho_k}\right)}{\beta},\\
	=& \sum_k \lambda_k \ip{\alpha}{\rho_k}\ip{\rho_k}{\beta}
\end{align}
\subsubsection{}
The eigenvalues all sum to unity:
\begin{align}
	\ip{\Psi} = \Tr(\rho) = 1 =& \sum_n \ev{\left(\sum_k \lambda_k \op{\rho_k}\right)}{n},\\
	=& \sum_n \sum_k \lambda_k \ip{n}{\rho_k}\ip{\rho_k}{n},\\
	=& \sum_n \sum_k \lambda_k \ip{\rho_k}{n}\ip{n}{\rho_k},\\
	=& \sum_k \lambda_k \ip{\rho_k},\\
	=& \sum_k \lambda_k.
\end{align}

The eigenvalues are between zero and one:
\begin{align}
	\forall \ket{\psi}, \ev{\rho}{\psi} =& \sum_{k} \lambda_k \abs{\ip{\psi}{\rho_k}}^2 \geq 0.\\
	\therefore \ev{\rho}{\rho_n} =& \sum_k \lambda_k \abs{\ip{\rho_n}{\rho_k}}^2,\\
	=& \lambda_n \geq 0.
\end{align}
Additionally, since the sum of eigenvalues is equal to unity, all eigenvalues must be less than one.

\printBib


\if\mcols1
\end{multicols*}
\fi
\end{document}
