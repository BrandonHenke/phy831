\documentclass[a4paper,12pt,twoside]{article}

\immediate\write18{echo -n "\\newcommand{\\gitroot}{" > gitroot.txt && git rev-parse --show-toplevel >> gitroot.txt && truncate -s-1 gitroot.txt && echo -n "}" >> gitroot.txt}

\input{gitroot.txt}
\usepackage{\gitroot/latexTemplate/preamble}


\newcommand{\Author}{Brandon Henke}
\newcommand{\course}{\textit{PHY831}}
\newcommand{\professor}{\textit{Scott Bogner}}

\newcommand{\mcols}{0}


\title{Homework 03}
\author{
	Brandon Henke\\
	\course\\
	\professor
}
\date{Ottobre 4, 2021}


\fancyhead[LE,RO]{B. Henke}
\fancyhead[RE,LO]{\thepage}

\bibSetup{refs.bib}

\begin{document}
%\tableofcontents

\maketitle
\if\mcols1
\begin{multicols*}{2}
\fi
\setcounter{section}{3}
\subsection{}
\subsubsection{}
For some polymer of length $l = 2\abs{s}\rho$, there are $N_+$ lengths going right and $N_-$ lengths going left, such that $N_+ - N_- = s$.
This is the same problem as the paramagnet problem, where the length of the chain is taking the place of the energy ($U=-2Bs$).
However, in the case of the polymer the length is the same in both the $\pm s$ cases.
Hence
\begin{equation}
	\Omega = \Omega(N,s)+\Omega(N,-s) = 2\Omega(N,s) = \frac{2N!}{(N/2+s)!(N/2-s)!}.
\end{equation}
\subsubsection{}
The entropy, $S$, is given by
\begin{align}
	S =& \log(\Omega),\\
	=& \log(2N!) - \log(N/2+s)! - \log(N/2-s)!.
\end{align}
Using Stirling's approximation on each term gives
\begin{align}
	\log(2N!) \approx& \log{2} + N\log{N} - N.\\
	\log(N/2+s)! \approx& (N/2+s)\log(N/2+s) - \frac{N}{2}-s.\\
	\log(N/2+s)! \approx& (N/2-s)\log(N/2-s) - \frac{N}{2}+s.\\
\end{align}
Thus, the entropy can be approximated as
\begin{align}
	S \approx& \log{2} + N\log{N} - N\nonumber\\
	&- (N/2+s)\log(N/2+s) + \frac{N}{2}-s\nonumber\\
	&- (N/2-s)\log(N/2-s) + \frac{N}{2}-s,\\
	=& \log{2} + N\log{N} - 2s - (N/2+s)\log(N/2+s)\nonumber\\
	&- (N/2-s)\log(N/2-s).
\end{align}
If $\abs{s} \ll N$, then (problem wants an $\abs{s}^2$ term, expand to an $s^2$ term):
\begin{align}
	(N/2+s)\log(N/2+s) \approx& \frac{N}{2}\log(N/2)+s\log(N/2)+s+\frac{s^2}{N},\\
	(N/2-s)\log(N/2-s) \approx& \frac{N}{2}\log(N/2)-s\log(N/2)-s+\frac{s^2}{N}.
\end{align}
Plugging these into the expression for the entropy gives
\begin{align}
	S \approx& \log{2} + N\log{N} - N\log(N/2) - \frac{2s^2}{N},\\
	=& \log(2\Omega(N,0)) - \frac{l^2}{2N\rho^2}.
\end{align}

\subsubsection{}
The force for a given length $\ell$ is given by
\begin{align}
	f(\ell) =& -T\eval{\pdv{S}{l}}_\ell,\\
	=& \frac{T\ell}{N\rho^2}.
\end{align}

\subsection{}
\subsubsection{}
The partition function, by definition, for one oscillator is given by
\begin{equation}
	Z_1 = \sum_m e^{-\beta \epsilon_m} = \sum_m e^{-\beta m\hbar \omega}.
\end{equation}
This is a geometric series, so
\begin{equation}
	Z_1 = \frac{1}{1-e^{-\beta \hbar \omega}}.
\end{equation}
The average energy is given by
\begin{align}
	\ev{U}_1 =& \frac{1}{Z_1} \sum_m \epsilon_m e^{-\beta \epsilon_m},\\
	=& -\pdv{\beta} \log Z_1,\\
	=& \frac{\hbar\omega}{e^{\beta\hbar\omega}-1}.
\end{align}
The Helmholtz free energy is given by
\begin{align}
	F =& - \frac{1}{\beta}\log Z_1,\\
	=& \frac{1}{\beta} \log(1-e^{-\beta\hbar\omega}).
\end{align}
The heat capacity at constant volume is given by
\begin{align}
	C_v =& \pdv{\ev{U}_1}{T},\\
	=& \frac{(\beta\hbar\omega)^2 e^{\beta\hbar\omega}}{(e^{\beta\hbar\omega}-1)^2}
\end{align}

\printBib


\if\mcols1
\end{multicols*}
\fi
\end{document}
