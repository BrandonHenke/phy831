\documentclass[a4paper,12pt,twoside]{article}

\immediate\write18{echo -n "\\newcommand{\\gitroot}{" > gitroot.txt && git rev-parse --show-toplevel >> gitroot.txt && truncate -s-1 gitroot.txt && echo -n "}" >> gitroot.txt}

\input{gitroot.txt}
\usepackage{\gitroot/latexTemplate/preamble}


\newcommand{\Author}{Brandon Henke}
\newcommand{\course}{\textit{PHY831}}
\newcommand{\professor}{\textit{Scott Bogner}}

\newcommand{\mcols}{0}


\title{Homework 03}
\author{
	Brandon Henke\\
	\course\\
	\professor
}
\date{Ottobre 4, 2021}


\fancyhead[LE,RO]{B. Henke}
\fancyhead[RE,LO]{\thepage}

\bibSetup{refs.bib}

\begin{document}
%\tableofcontents

\maketitle
\if\mcols1
\begin{multicols*}{2}
\fi
\setcounter{section}{3}
\subsection{}
\subsubsection{}
For some polymer of length $l = 2\abs{s}\rho$, there are $N_+$ lengths going right and $N_-$ lengths going left, such that $N_+ - N_- = s$.
This is the same problem as the paramagnet problem, where the length of the chain is taking the place of the energy ($U=-2Bs$).
However, in the case of the polymer the length is the same in both the $\pm s$ cases.
Hence
\begin{equation}
	\Omega = \Omega(N,s)+\Omega(N,-s) = 2\Omega(N,s) = \frac{2N!}{(N/2+s)!(N/2-s)!}.
\end{equation}
\subsubsection{}
The entropy, $S$, is given by
\begin{align}
	S =& \log(\Omega),\\
	=& \log(2N!) - \log(N/2+s)! - \log(N/2-s)!.
\end{align}
Using Stirling's approximation on each term gives
\begin{align}
	\log(2N!) \approx& \log{2} + N\log{N} - N.\\
	\log(N/2+s)! \approx& (N/2+s)\log(N/2+s) - \frac{N}{2}-s.\\
	\log(N/2+s)! \approx& (N/2-s)\log(N/2-s) - \frac{N}{2}+s.\\
\end{align}
Thus, the entropy can be approximated as
\begin{align}
	S \approx& \log{2} + N\log{N} - N\nonumber\\
	&- (N/2+s)\log(N/2+s) + \frac{N}{2}-s\nonumber\\
	&- (N/2-s)\log(N/2-s) + \frac{N}{2}-s,\\
	=& \log{2} + N\log{N} - 2s - (N/2+s)\log(N/2+s)\nonumber\\
	&- (N/2-s)\log(N/2-s).
\end{align}
If $\abs{s} \ll N$, then (problem wants an $\abs{s}^2$ term, expand to an $s^2$ term):
\begin{align}
	(N/2+s)\log(N/2+s) \approx& \frac{N}{2}\log(N/2)+s\log(N/2)+s+\frac{s^2}{N},\\
	(N/2-s)\log(N/2-s) \approx& \frac{N}{2}\log(N/2)-s\log(N/2)-s+\frac{s^2}{N}.
\end{align}
Plugging these into the expression for the entropy gives
\begin{align}
	S \approx& \log{2} + N\log{N} - N\log(N/2) - \frac{2s^2}{N},\\
	=& \log(2\Omega(N,0)) - \frac{l^2}{2N\rho^2}.
\end{align}

\subsubsection{}
The force for a given length $\ell$ is given by
\begin{align}
	f(\ell) =& -T\eval{\pdv{S}{l}}_\ell,\\
	=& \frac{T\ell}{N\rho^2}.
\end{align}

\subsection{}
\subsubsection{}
The partition function, by definition, for one oscillator is given by
\begin{equation}
	Z_1 = \sum_m e^{-\beta \epsilon_m} = \sum_m e^{-\beta m\hbar \omega}.
\end{equation}
This is a geometric series, so
\begin{equation}
	Z_1 = \frac{1}{1-e^{-\beta \hbar \omega}}.
\end{equation}
The average energy is given by
\begin{align}
	\ev{U}_1 =& -\frac{1}{Z_1}\pdv{Z_1}{\beta},\\
	=& \frac{\hbar\omega}{e^{\beta\hbar\omega}-1}.
\end{align}
The Helmholtz free energy is given by
\begin{align}
	F_1 =& - \frac{1}{\beta}\log Z_1,\\
	=& \frac{1}{\beta} \log(1-e^{-\beta\hbar\omega}).
\end{align}
The heat capacity at constant volume is given by
\begin{align}
	C_{v,1} =& \pdv{\ev{U}_1}{T},\\
	=& \frac{(\beta\hbar\omega)^2 e^{\beta\hbar\omega}}{(e^{\beta\hbar\omega}-1)^2}
\end{align}

\subsubsection{}
For a classical system, the partition function is given by
\begin{equation}
	Z = \frac{1}{(2\pi\hbar)^3}\int e^{-\beta H(q,p)} \dd[3]{q}\dd[3]{p}.
\end{equation}
In the case of a single harmonic oscillator,
\begin{equation}
	H(q,p) = \frac{p^2}{2m} + \frac{1}{2}m\omega q^2.
\end{equation}
Itaque,
\begin{align}
	Z =& \frac{1}{(2\pi\hbar)^3}\int e^{-\frac{\beta p^2}{2m}}\dd[3]{p}\int e^{-\frac{1}{2}\beta m \omega q^2}\dd[3]{q},\\
	=& \frac{16\pi^2}{(2\pi\hbar)^3} \frac{\pi}{2} \frac{1}{\beta^3 \omega^{3/2}},\\
	=& \frac{1}{(\beta\hbar\sqrt{\omega})^{3}}.
\end{align}
In ultra, the average energy is given by
\begin{align}
	\ev{U} =& -\frac{1}{Z} \pdv{Z}{\beta},\\
	=& (\beta\hbar\sqrt{\omega})^{3}\frac{3}{\beta^4(\hbar\sqrt{\omega})^{3}},\\
	=& \frac{3}{\beta}.
\end{align}

In the limit where $\hbar \rightarrow 0$ the average energy for the quantum case becomes that of classical case for an oscillator in one dimension.
The classical case was done for three dimensions, which is why there is a factor of three.

\subsubsection{}
For two non-interacting systems at the same temperature, the same energy levels are possible in each system.
Itaque, the partition function is given by
\begin{align}
	Z(M+N) =& \sum_{m}\sum_{n} e^{-\beta (\epsilon_m+\epsilon_n)},\\
	=& \sum_{m}\sum_{n} e^{-\beta\epsilon_m}e^{-\beta\epsilon_n},\\
	=& \left(\sum_{m}e^{-\beta\epsilon_m}\right)\left(\sum_{n}e^{-\beta\epsilon_n}\right),\\
	=& Z(M)Z(N).
\end{align}

\subsubsection{}
For a system of $N$ oscillators, the partition function becomes
\begin{equation}
	Z_N = Z_1^N = \frac{1}{\left(1-e^{-\beta \hbar \omega}\right)^N}.
\end{equation}

The average energy is
\begin{equation}
	\ev{U}_N = -\frac{1}{Z_1^N} \pdv{Z_1^N}{\beta} = -\frac{N}{Z_1} \pdv{Z_1}{\beta} = N\ev{U}_1 = \frac{N\hbar\omega}{e^{\beta\hbar\omega}-1}.
\end{equation}

The Helmholtz free energy is
\begin{equation}
	F_N = -\frac{1}{\beta}\log{Z} = -\frac{N}{\beta}\log{Z_1} = NF_1 = \frac{N}{\beta} \log(1-e^{-\beta\hbar\omega})
\end{equation}

The Heat capacity at constant volume is
\begin{equation}
	C_{v,N} = \pdv{\ev{U}}{T} = N\pdv{\ev{U}_1}{T} = N C_{v,1} =  \frac{N(\beta\hbar\omega^2)e^{\beta\hbar\omega}}{(e^{\beta\hbar\omega}-1)^2}.
\end{equation}

\subsection{}

The partition function of the state is given by
\begin{align}
	Z(T,N,V,B) =& \prod_n \left(\int e^{-\frac{\beta p_n^2}{2m}}\dd[3]{p}\right)\left(\sum_m e^{\beta \mu B S_{n,z}}\right),\\
	=& \prod_n \left(\frac{2\pi m}{\beta}\right)^{3/2}\left( Stuff? \right)
\end{align}


\printBib


\if\mcols1
\end{multicols*}
\fi
\end{document}
