\documentclass[a4paper,12pt,twoside]{article}

\immediate\write18{echo -n "\\newcommand{\\gitroot}{" > gitroot.txt && git rev-parse --show-toplevel >> gitroot.txt && truncate -s-1 gitroot.txt && echo -n "}" >> gitroot.txt}

\input{gitroot.txt}
\usepackage{\gitroot/latexTemplate/preamble}


\newcommand{\Author}{Brandon Henke}
\newcommand{\course}{PHY831}
\newcommand{\professor}{Scott Bogner}

\newcommand{\mcols}{0}


\title{Homework 5}
\author{
	Brandon Henke\\
	\textit{\course}\\
	\textit{\professor}
}
\date{Novembre 2, 2021}


\fancyhead[LE,RO]{B. Henke}
\fancyhead[RE,LO]{\thepage}

\bibSetup{refs.bib}

\begin{document}
%\tableofcontents

\maketitle
\if\mcols1
\begin{multicols*}{2}
\fi

\setcounter{section}{5}
\subsection{}
\subsubsection{}
\begin{align}
	-\frac{\hbar^2}{2I}\psi_n''(\theta) =& E_n \psi_n(\theta),\\
	\rightarrow& \psi_n(\theta) = c_1 \cos(\frac{\sqrt{2IE_n}}{\hbar}\theta) + c_2 \sin(\frac{\sqrt{2IE_n}}{\hbar}\theta).
\end{align}
From boundary conditions,
\begin{align}
	\cos(\frac{\sqrt{2IE_n}}{\hbar}2\pi) =& 1,\\
	\frac{\sqrt{2IE_n}}{\hbar}2\pi =& 2n\pi,\\
	E_n =& \frac{n^2\hbar^2}{2I}.
\end{align}
\subsubsection{}


\subsection{}
Find grand partition function, then grand potential, then pressure from grand potential.
There will be an integration by parts.
\subsubsection{}
\subsubsection{}
\subsubsection{}
\subsection{}
See lectures 20 and 21 for density of states.
\subsection{}
See lectures 20 and 21 for summerfeld expansion.
\subsection{}

\printBib


\if\mcols1
\end{multicols*}
\fi
\end{document}
