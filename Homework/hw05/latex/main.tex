\documentclass[a4paper,12pt,twoside]{article}

\immediate\write18{echo -n "\\newcommand{\\gitroot}{" > gitroot.txt && git rev-parse --show-toplevel >> gitroot.txt && truncate -s-1 gitroot.txt && echo -n "}" >> gitroot.txt}

\input{gitroot.txt}
\usepackage{\gitroot/latexTemplate/preamble}


\newcommand{\Author}{Brandon Henke}
\newcommand{\course}{PHY831}
\newcommand{\professor}{Scott Bogner}

\newcommand{\mcols}{0}


\title{Homework 5}
\author{
	Brandon Henke\\
	\textit{\course}\\
	\textit{\professor}
}
\date{Novembre 2, 2021}


\fancyhead[LE,RO]{B. Henke}
\fancyhead[RE,LO]{\thepage}

\bibSetup{refs.bib}

\begin{document}
%\tableofcontents

\maketitle
\if\mcols1
\begin{multicols*}{2}
\fi

\setcounter{section}{5}
\subsection{}
\subsubsection{}
\begin{align}
	-\frac{\hbar^2}{2I}\psi_n''(\theta) =& E_n \psi_n(\theta),\\
	\therefore \psi_n(\theta) =& c_{1,n} e^{ik_n\theta}+c_{2,n} e^{-ik_n\theta},
	\label{eq: 5.1 gen_sol}
\end{align}
where $ k_n = \sqrt{2IE_n}/\hbar$.
Since $\comm{H}{L_z}=0$ the solution to Shr\"odinger's equation must also be an eigenstate of the $L_z$ operator:
\begin{align}
	\hat{L}_z \ket{\psi_n} =& l_n \ket{\psi_n},\\
	-i\hbar\pdv{\psi_n}{\theta} =& l_n \psi_n,\\
	\psi_n (\theta) =& d_n e^{i l_n \theta/\hbar}.
\end{align}
Therefore, in equation \ref{eq: 5.1 gen_sol}, either $c_{1,n} = 0$ or $c_{2,n} = 0$, so, let $c_{2,n} = 0$:
\begin{equation}
	\psi_n(\theta) = c_{1,n} e^{i k_n \theta}.
	\label{eq: 5.1 sol}
\end{equation}
Since the boundary conditions are periodic, with a period of $2\pi$,
\begin{align}
	\frac{\sqrt{2IE_n}}{\hbar} \in \mathbb{Z}\rightarrow n =& \frac{\sqrt{2IE_n}}{\hbar},\\
	E_n =& \frac{n^2\hbar^2}{2I}.
\end{align}
Additionally, $\abs{\psi_n}^2 = 1$, so
\begin{align}
	1 =& \int_0^{2\pi} c_n^2 \dd{\theta},\\
	c_n =& \sqrt{\frac{1}{2\pi}}.
\end{align}

Therefore,
\begin{align}
	\psi_n(\theta) =& \frac{1}{\sqrt{2\pi}}e^{i n\theta},\\
	E_n =& \frac{n^2\hbar^2}{2I}.
\end{align}

\subsubsection{}
Let $\rho = \frac{1}{Z} e^{-\beta\hat{H}}$, where $Z$ is the canonical partition function, and $\hat{H}$ is the Hamiltonian.
Then
\begin{align}
	\mel{\theta'}{\rho}{\theta} =& \frac{1}{Z} \mel{\theta'}{e^{-\beta\hat{H}}}{\theta},\\
	=& \frac{\sum_n e^{-\beta E_n}\ip{\theta'}{\psi_n}\ip{\psi_n}{\theta}}{\Tr( e^{-\beta \hat{H}})},\\
	=& \frac{\sum_n e^{-\beta E_n}e^{i n(\theta'-\theta)}}{2\pi\Tr(e^{-\beta \hat{H}})}.
\end{align}

In the high temperature limit, $\beta \rightarrow 0$, and the sum can be closely approximated by an integral:
\begin{align}
	\mel{\theta'}{\rho}{\theta}
	=& \sqrt{\frac{\beta\hbar^2}{2\pi I}}\frac{1}{2\pi} \int e^{-\beta E_n}e^{i n(\theta'-\theta)}\dd{n},\\
	=& \sqrt{\frac{\beta\hbar^2}{2\pi I}}\frac{1}{2\pi} \int e^{-\beta \frac{n^2\hbar^2}{2I}+in\Delta\theta}\dd{n},\\
	=& \sqrt{\frac{\beta\hbar^2}{2\pi I}}\frac{1}{2\pi} \int e^{-\beta \frac{n^2\hbar^2}{2I}+in\Delta\theta}\dd{n}
\end{align}

\subsection{}
\subsubsection{}
\subsubsection{}
\subsubsection{}
\subsection{}
Find grand partition function, then grand potential, then pressure from grand potential.
There will be an integration by parts.
\subsection{}
See lectures 20 and 21 for density of states.
\subsection{}
See lectures 20 and 21 for summerfeld expansion.

\printBib


\if\mcols1
\end{multicols*}
\fi
\end{document}
