\documentclass[a4paper,12pt,twoside]{article}

\immediate\write18{echo -n "\\newcommand{\\gitroot}{" > gitroot.txt && git rev-parse --show-toplevel >> gitroot.txt && truncate -s-1 gitroot.txt && echo -n "}" >> gitroot.txt}

\input{gitroot.txt}
\usepackage{\gitroot/latexTemplate/preamble}


\newcommand{\Author}{Brandon Henke}
\newcommand{\course}{PHY831}
\newcommand{\professor}{Scott Bogner}

\newcommand{\mcols}{0}


\title{Homework 5}
\author{
	Brandon Henke\\
	\textit{\course}\\
	\textit{\professor}
}
\date{Novembre 2, 2021}


\fancyhead[LE,RO]{B. Henke}
\fancyhead[RE,LO]{\thepage}

\bibSetup{refs.bib}

\begin{document}
%\tableofcontents

\maketitle
\if\mcols1
\begin{multicols*}{2}
\fi

\setcounter{section}{5}
\subsection{}
\subsubsection{}
\begin{align}
	-\frac{\hbar^2}{2I}\psi_n''(\theta) =& E_n \psi_n(\theta),\\
	\therefore \psi_n(\theta) =& c_{1,n} e^{ik_n\theta}+c_{2,n} e^{-ik_n\theta},
	\label{eq: 5.1 gen_sol}
\end{align}
where $ k_n = \sqrt{2IE_n}/\hbar$.
Since $\comm{H}{L_z}=0$ the solution to Shr\"odinger's equation must also be an eigenstate of the $L_z$ operator:
\begin{align}
	\hat{L}_z \ket{\psi_n} =& l_n \ket{\psi_n},\\
	-i\hbar\pdv{\psi_n}{\theta} =& l_n \psi_n,\\
	\psi_n (\theta) =& d_n e^{i l_n \theta/\hbar}.
\end{align}
Therefore, in equation \ref{eq: 5.1 gen_sol}, either $c_{1,n} = 0$ or $c_{2,n} = 0$, so, let $c_{2,n} = 0$:
\begin{equation}
	\psi_n(\theta) = c_{1,n} e^{i k_n \theta}.
	\label{eq: 5.1 sol}
\end{equation}
Since the boundary conditions are periodic, with a period of $2\pi$,
\begin{align}
	\frac{\sqrt{2IE_n}}{\hbar} \in \mathbb{Z}\rightarrow n =& \frac{\sqrt{2IE_n}}{\hbar},\\
	E_n =& \frac{n^2\hbar^2}{2I}.
\end{align}
Additionally, $\abs{\psi_n}^2 = 1$, so
\begin{align}
	1 =& \int_0^{2\pi} c_n^2 \dd{\theta},\\
	c_n =& \sqrt{\frac{1}{2\pi}}.
\end{align}

Therefore,
\begin{align}
	\psi_n(\theta) =& \frac{1}{\sqrt{2\pi}}e^{i n\theta},\\
	E_n =& \frac{n^2\hbar^2}{2I}.
\end{align}

\subsubsection{}
Let $\rho = \frac{1}{Z} e^{-\beta\hat{H}}$, where $Z$ is the canonical partition function, and $\hat{H}$ is the Hamiltonian.
Then
\begin{align}
	\mel{\theta'}{\rho}{\theta} =& \frac{1}{Z} \mel{\theta'}{e^{-\beta\hat{H}}}{\theta},\\
	=& \frac{\sum_n e^{-\beta E_n}\ip{\theta'}{\psi_n}\ip{\psi_n}{\theta}}{\Tr( e^{-\beta \hat{H}})},\\
	=& \frac{\sum_n e^{-\beta E_n}e^{i n(\theta'-\theta)}}{2\pi\Tr(e^{-\beta \hat{H}})}.
\end{align}

In the high temperature limit, $\beta \rightarrow 0$, and the sum can be closely approximated by an integral:
\begin{align}
	\mel{\theta'}{\rho}{\theta}
	=&\sqrt{\frac{\beta\hbar^2}{2\pi I}}\frac{1}{2\pi} \int e^{-\beta E_n}e^{i n(\theta'-\theta)}\dd{n},\\
	=&\sqrt{\frac{\beta\hbar^2}{2\pi I}}\frac{1}{2\pi} \int e^{-\beta \frac{n^2\hbar^2}{2I}+in\Delta\theta}\dd{n},\\
	=&\sqrt{\frac{\beta\hbar^2}{2\pi I}}\frac{1}{2\pi} \int e^{-\beta \frac{n^2\hbar^2}{2I}+in\Delta\theta}\dd{n},\\
	=&\sqrt{\frac{\beta\hbar^2}{2\pi I}}\frac{1}{2\pi} \sqrt{\frac{2\pi I}{\beta \hbar^2}} e^{-\frac{\Delta\theta^2 I}{2 \beta \hbar^2}},\\
	=&\frac{1}{2\pi} e^{-\frac{\Delta\theta^2 I}{2\beta\hbar^2}}.
\end{align}

In the low temperature limit, $\beta \rightarrow \infty$, so the lower energy states dominate:
\begin{align}
	\mel{\theta'}{\rho}{\theta}
	=& \frac{\sum_n e^{-\beta E_n}e^{i n(\theta'-\theta)}}{2\pi\Tr(e^{-\beta \hat{H}})},\\
	\approx \frac{1}{2\pi} \frac{1}{1},\\
	= \frac{1}{2\pi}.
\end{align}

\subsection{}
For $N$-identical particles, the partition function is given by
\begin{equation}
	Z_N (V,T) = \frac{1}{N!} \left(\frac{V^N}{l_Q^{3N}} + \int\prod_{a=1}^N \dd[3]{\vb{x}_a}\sum_{p=1}^{N!-1}\eta^p e^{-\frac{\pi}{l_Q^2}\sum_{a=1}^N (\vb{x}_a-\vb{x}_{p_a})^2}\right),
\end{equation}
where $\eta = 1$ for bosons and $\eta = -1$ for fermions.
\subsubsection{}
For $N=2$,
\begin{align}
	Z_2 =& \frac{1}{2} \left( \frac{V^2}{l_Q^6} + \int \prod_{a=1}^2 \dd[3]{x_a} \sum_{p = 1}^1 \eta^p e^{-\frac{\pi}{l_Q^2}\sum_{a=1}^2 (\vb{x}_a-\vb{x}_{p_a})^2} \right),\\
	=& \frac{1}{2} \left( \frac{V^2}{l_Q^6} + \eta\int \dd[3]{x_1}\dd[3]{x_2} e^{-\frac{2\pi}{l_Q^2}(\vb{x}_1-\vb{x}_2)^2} \right),\\
	=& \frac{1}{2} \left( \frac{V^2}{l_Q^6} \pm \frac{V}{2^{3/2} l_Q^3} \right). \qquad \eta = \left\{
		\begin{array}{rl}
			+1, & \text{for bosons}\\
			-1, & \text{for fermions}\\
		\end{array}
	\right.
\end{align}
Since $Z_1(m) = V/l_Q^3$, then $Z_1(m/2) = V/(2^{3/2} l_Q^3)$.
Thus
\begin{equation}
	Z_2(m) = \frac{1}{2} \left(Z_1^2(m) \pm Z_1(m/2)\right).
\end{equation}
\subsubsection{}
The average energy is given by
\begin{equation}
	\ev{E} = \frac{1}{Z_N} \sum_n \epsilon_n e^{-\beta \epsilon_n}.
\end{equation}
One can see that one can acheive this result by taking the negative of the derivative of the partition function with respect to $\beta$, then dividing by the partition function.
Thus, for this problem, the average energy is given by
\begin{align}
	\ev{E} =& -\frac{1}{Z_2} \pdv{Z_2}{\beta},\\
	=& -\pdv{\beta} \log(Z_2),\\
	=& -\pdv{\beta} \log(\frac{Z_1(m)^2}{2}) \pm \frac{Z_1(m/2)}{Z_1^2(m)},\\
	=& \ev{E}_\sim + \delta\ev{E},
\end{align}
where
\begin{equation}
	\delta\ev{E} = \mp \pdv{\beta} \frac{Z_1(m/2)}{Z_1^2(m)} = \mp \frac{3}{2^{5/2}} \frac{l_Q^3}{V}k_B T.
\end{equation}

Additionally
\begin{equation}
	\delta C_v = \pdv{T}\delta\ev{E} = \mp \frac{3}{2^{5/2}} \frac{l_Q^3}{V}k_B.
\end{equation}

\subsubsection{}
For the corrections to be small,
\begin{equation}
	Z_1^2(m) \gg Z_1(m/2) \rightarrow V \gg \frac{l_Q^3}{2^{3/2}}.
\end{equation}
If this is not the case, $l_Q \rightarrow V^{1/3} = L$.
Hence $T < \frac{2\pi \hbar^2}{m k_B L^2}$.

\subsection{}
\begin{align}
	PV\beta = \log(\mathcal{Z}) =& -\eta(2S+1)V\int\frac{\dd[3]{p}}{(2\pi\hbar)^3}\log(1-\eta e^{-\beta(\epsilon(p) - \mu)}),\\
	=& -\frac{\eta(2S+1)V}{2\pi^2\hbar^3} \int \dd{p} p^2 \log(1-\eta e^{-\beta(\epsilon(p)-\mu)}),\\
\end{align}
\subsection{}
See lectures 20 and 21 for density of states.
\subsection{}
See lectures 20 and 21 for summerfeld expansion.

\printBib


\if\mcols1
\end{multicols*}
\fi
\end{document}
